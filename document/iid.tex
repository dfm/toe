\documentclass[12pt, preprint]{aastex}

\newcommand{\documentname}{\textsl{Note}}

\renewcommand{\d}{\mathrm{d}}

\begin{document}

\title{IID: How is the architecture of exoplanetary systems inconsistent with statistical independence?}
\author{
  some combination of:
  Daniel~Foreman-Mackey\altaffilmark{1, 2},
  Patrick Cooper\altaffilmark{1},
  Ross~Fadely\altaffilmark{1},
  David~W.~Hogg\altaffilmark{1}
  Meg~Schwamb\altaffilmark{3},
  Julian~Taub\altaffilmark{4}
}
\altaffiltext{1}{Center for Cosmology and Particle Physics, New York University}
\altaffiltext{2}{to whom correspondence should be addressed}
\altaffiltext{3}{Department of Astronomy, Yale University}
\altaffiltext{4}{Freelance!}

\begin{abstract}
  The mass, chemical composition, and age of a star are all
  undoubtedly related to the probability that the star will host an
  exoplanetary system, and exoplanets within a single extrasolar
  system must interact gravitationally.  Nonetheless, in this
  \documentname\ we consider a trivial model for exoplanetary systems
  in which exoplanets are assigned to stars \emph{independently}, with
  no regard to stellar properties or the exisitence or properties of
  other exoplanets in the system.  Although there is freedom in choice
  of quantities to be considered ``independent'', we find that the
  model is not a good fit to the data for any simple choices.  In
  particular the independence asumption is violated in the following
  respects: blah, blah blah.
\end{abstract}

\keywords{
  methods: statistical
  ---
  celestial mechanics
  ---
  planetary systems
  ---
  planets and satellites: dynamical evolution and stability
  ---
  planets and satellites: fundamental parameters
  ---
  surveys
}

\section{Introduction}

...Independence; why it cannot be true.  Why it is interesting to study....

...Here are things that \emph{are} known about the exoplanet population:
\begin{itemize}
\item
Accounting for selection effects, the distribution of transiting
planet radii $R_p$ seems to be $\d N / \d\ln R_p\propto R^{-2}$,
or $\d N/ \d R_p\propto R^{-3}$ (\citealt{howard}, \citealt{youdin}).
\item
The distribution of multi-planet transit systems with planet periods
less than 200~d is best explained by a small but finite inclination
distribution.  The best-fit inclination distributions have scatters of
1 to 3~deg, suggesting strong but not exact coplanarity, and mirroring
our own solar system (\citealt{fang}).
\end{itemize}

\section{Causal model}

Overall, the model is...

Specifically, we model...  

\section{Data and results}

\section{Discussion}

\acknowledgements Big thanks to Gus Muench (CfA) who organized the
dotastronomy NYC Hack Day on 2012 December 15, where this project was
started.  It is a pleasure to also thank....and acknowledge....etc.

\begin{thebibliography}{70}
\raggedright
\bibitem[Fang \& Margot(2012)]{fang}
Fang, J., \& Margot, J.-L.\ 2012, \apj, 761, 92
\bibitem[Howard et al.(2012)]{howard}
Howard, A.~W., Marcy, G.~W., Bryson, S.~T., et al.\ 2012, \apjs, 201, 15
\bibitem[Youdin(2011)]{youdin}
Youdin, A.~N.\ 2011, \apj, 742, 38
\end{thebibliography}

\end{document}
